\documentclass[12pt]{article}

\usepackage{sbc-template}

\usepackage{graphicx,url}

\usepackage{mathtools}

\usepackage[brazil]{babel}   
\usepackage[utf8]{inputenc}  

\usepackage[autostyle]{csquotes}  

\usepackage{multirow}
\usepackage[table,xcdraw]{xcolor}
     
\sloppy

\title{Sistema de Recomendação Baseado em Confiança para Promover a Colaboração em Redes de Pesquisa Científica}

\author{João Pedro R. D. Saldanha\inst{1}, Fernando Prass\inst{1}}


\address{Ciência da Computação -- Universidade Franciscana (UFN)\\
  Rua dos Andradas, 1614  -- 97010-032  -- Santa Maria -- RS -- Brasil
\email{\{joao.pedro,fernando.prass\}@ufn.edu.br}
}

\begin{document} 

\maketitle

\begin{abstract}
  This paper presents a proposal for building a recommender system that promotes collaboration between researchers in a 
  publication network. A trust network is abstracted in which researchers are bound by joint publications, which represents 
  a mutual trust statement. Three metrics are discussed for  computing trust: global trust (from the community’s perspective), 
  local trust (subjective to each researcher) or via the concept of distances, seeking “friends of a friend”. An architecture 
  is proposed to improve recommendation quality trough profile analysis, as well as validation metrics.
\end{abstract}
     
\begin{resumo} 
  Este artigo apresenta a proposta da elaboração de um sistema de recomendações para promover a colaboração entre pesquisadores 
  em uma rede de publicações. É abstraída uma rede de confiança na qual pesquisadores são unidos por publicações em conjunto, 
  que constituem um voto de confiança mútua. São discutidas três métricas de computação de confiança: confiança global (da 
  comunidade como um todo), local (subjetivo a cada pesquisador) ou com o conceito de distâncias, buscando “amigos de amigos”. 
  Foi proposta uma arquitetura para melhorar as recomendações através de análise  de perfil, além de métricas de validação.
\end{resumo}


\section{Introdução}

No método científico, pesquisadores devem realizar um trabalho criativo sistemático para incrementar 
o conhecimento na área da pesquisa. Parte importante do processo é a busca, dentro do universo dos 
trabalhos científicos, por embasamento teórico ao tema do trabalho proposto. Além disto, também é 
relevante conhecer e colaborar com outros pesquisadores desenvolvendo trabalhos relacionados dentro da 
área de pesquisa.  Portanto, é preciso reunir todas as informações pertinentes, pesquisas e resultados 
anteriores bem como linhas de pesquisa em progresso para não reinventar a roda ou seguir caminhos já 
trilhados e assim realizar trabalho relevante e produtivo.

A plataforma Lattes é um sistema que integra bases de dados de currículos, em específico de pesquisadores. 
Ela oferece aos usuários a possibilidade de criar um currículo de maneira gratuita, que é disponibilizado 
abertamente aos visitantes do site. Segundo \cite{CNPq2019lattes}, o Currículo Lattes 
\textit{"se tornou um padrão nacional no registro da vida pregressa e atual dos estudantes e 
pesquisadores do país, e é hoje adotado pela maioria das instituições de fomento, universidades e 
institutos de pesquisa do País"}.
    
O universo da pesquisa científica está em constante expansão, tanto no que diz respeito ao conhecimento 
produzido quanto ao volume de trabalhos e publicações. Estimativas apontavam um valor em torno de 2.5 
milhões de artigos científicos publicados por ano, em 2015, com um aumento de 5\% ao ano no número de 
cientistas fazendo publicações \cite{ware2015stm}. Pesquisadores não têm o tempo necessário para analisar 
todos os estudos relacionados à seus próprios trabalhos, mesmo com plataformas como o Lattes, onde 
tais trabalhos estão compilados. Trata-se do problema da sobrecarga de informação, que tem crescido na 
medida em que sistemas digitais vem ganhando cada vez usuários e conteúdo. Outro problema decorrente do 
crescimento do número de pesquisadores e trabalhos é que muitas vezes os pesquisadores não conhecem outros 
pesquisadores da área e acabam por perder a oportunidade de colaborações ou troca de ideias.

Logo, se faz necessário a filtragem da informação que chega ao pesquisador para maximizar sua eficiência e 
evitar tempo perdido. A automação da tarefa de filtragem é feita através de sistemas de recomendação. 
Utilizando técnicas de mineração de dados e inteligência artificial pode-se oferecer conteúdo mais relevante, 
aumentando a eficiência do acadêmico. 

A partir do problema da sobrecarga de informações, nos anos 90 iniciou-se a pesquisa na área de filtragem de conteúdo. 
O ponto de partida foi a observação que as pessoas usam, no dia-a-dia, dicas de outros para tomar decisões, sendo que 
as dicas daqueles tidos como especialistas no assunto tem um peso diferenciado. Os primeiros sistemas de recomendação 
eram algoritmos capazes de analizar tendências dentro de uma certa comunidade e então fazer sugestões aos seus membros. 
Este método é conhecido como filtragem colaborativa e foi aprimorado desde então, sendo até hoje bastante popular. Além 
deste, também é bastante difundido o método baseado em conteúdo, no qual novos itens são recomendados baseado no 
conteúdo consumido pelo usuário no passado \cite{ricci2011introduction}.

\subsection{Objetivos}

Este trabalho propõe a elaboração de um sistema de recomendações de pesquisadores e trabalhos científicos baseado 
em confiança a partir de dados da plataforma Lattes. Para tal, é preciso:

\begin{itemize}
  \item Estudar funções e aplicações de sistemas de recomendação
  \item Modelar a rede de confiança da comunidade científica
  \item Estabelecer métricas de confiança para os dados disponíveis
  \item Estimar a confiança entre os pesquisadores 
  \item Pré-selecionar recomendações
  \item Filtrar a pré-seleção com a confiança computada
  \item Avaliar o modelo
\end{itemize}

\subsection{Proposta}

A ideia é analisar o sentimento de confiança entre os pesquisadores e sugerir conteúdo relevante baseado em seus perfis, 
ponderando o conteúdo com base na confiança estimada. Para solucionar o problema explanado, propõe-se descrever uma rede de colaborações 
a partir de publicações em conjunto, discutida na seção \ref{sect:computing-trust} utilizando técnicas encontradas na literatura 
para computar a propagação de confiança na rede. Para isso, algumas métricas são discutidas na seção \ref{sect:computing-trust}. 
A partir daí, é discutida a pré-seleção dos itens com o método baseado em conteúdo (seção \ref{sect:pre-selection}), proposta uma 
arquitetura para o sistema de recomendações (seção \ref{sect:arch}) e uma metodologia para sua validação 
(seção \ref{sect:validation}).

\section{Revisão Bibliográfica}

Nesta seção, são apresentados os principais itens que compõem o embasamento teórico usado como ponto de 
partida no trabalho e usado para justificar as decisões tomadas.

\subsection{Sistemas de Recomendação}

Frequentemente usuários de plataformas digitais se deparam com situações nas quais é necessário escolher entre 
vários itens ofertados (produtos, conteúdo ou pessoas, por exemplo). A dificuldade de filtrar o conteúdo encontrado em determinada 
plataforma tende a aumentar na medida em que o número de itens ofertados cresce, visto que é necessário fazer uma 
análise individual de tais itens e então compará-los para fazer uma escolha. Para ajudar na tarefa, é comum encontrar 
sistemas que automatizam o processo de escolha, filtrando o conteúdo com base no perfil do usuário para apresentar 
itens de seu interesse. Tais sistemas são especificamente úteis quando um usuário encontra dificuldades para analisar os itens 
ofertados e fazer escolhas. Os itens recomendados pelo Sistema de Recomendação (SR) podem ser os mais variados, sendo que no geral a recomendação é uma tarefa especializada, ou seja, apenas um tipo de item é recomendado, e a recomendação é relevante para um perfil 
específico de usuário. Logo, as características do SR, como metodologia usada para sua construção, interface de usuário 
e critério para ordenar os resultados devem ser adaptados às especificidades da tarefa em questão \cite{ricci2011introduction}. 

A forma mais simples do resultado de um SR é uma lista de itens ordenada de acordo com a preferência do usuário. A satisfação 
com as recomendações pode ser coletada explicitamente, como por exemplo através de avaliações, ou implicitamente através de 
inferências baseadas no comportamento do usuário perante aos itens oferecidos. Para oferecer recomendações, é preciso analisar 
uma base de conhecimento, que pode ter diversas informações, realizar um trabalho de classificação dos itens ofertados e 
então coletar algum tipo de \textit{feedback} perante o resultado que deve ser usado para aprimorar o sistema \cite{shani2011evaluating}.

\subsubsection{Técnicas de Recomendação}

O resultado obtido por um SR é dependente da realização de uma \textbf{predição}. A predição é fundamental para a qualidade 
das recomendações: itens são apresentados ao usuário porque o sistema antecipa que sejam relevantes para ele 
\cite{ricci2011introduction}. Geralmente na elaboração de sistemas de recomendação lida-se com \textbf{usuários}, denotados por 
$ u_1, ... u_n \in U $, \textbf{itens}, denotados por $ i_1, ... i_n \in I$  e \textbf{relações}, que associam usuários e 
itens de diversas maneiras \cite{ekstrand2019recommender}. As associações podem ser representadas por ontologias 
\cite{primo2006tecnicas} ou no caso de relações entre usuários e itens através de uma matriz de associação $ |U| \times |I| $. 
Assume-se a existência no mundo real de uma \textbf{função} $ f (u, i) $ que retorne um número real representado a utilidade do 
item $i$ ao usuário $u$. Em técnicas de filtragem colaborativa, este numero é visto como a avaliação do usuário. A tarefa do SR 
neste contexto é computar uma função $\hat{f}(u, i)$ que se assemelhe ao máximo à $f$. 
Assim, é possível realizar a predição de relevância de um grupo de itens para determinado usuário $\hat{f}(u_n, I)$ e recomendar 
os itens melhores classificados pelo SR, efetivamente filtrando o conteúdo e oferecendo ao usuário uma seleção personalizada de 
itens \cite{ricci2011introduction}.

\subsubsection{Filtragem Colaborativa}

Técnicas de filtragem colaborativa analisam o \textbf{perfil} do usuário e sua \textbf{avaliação} dos itens previamente 
acessados para chegar em recomendações. Procura-se analisar o perfil do \textbf{usuário alvo} para então achar um \textit{cluster} 
de usuários com perfis similares (\textbf{vizinhos}). A ideia é que os itens bem avaliados pelos vizinhos serão também 
avaliados positivamente pelo usuário alvo, já que os perfis são semelhantes. Um problema encontrado na técnica é o da \textit{primeira avaliação}: quando há um item novo, sem nenhuma avaliação, como saber se determinado usuário irá avaliar 
positivamente o mesmo? Nenhum de seus vizinhos fez avaliações \cite{ricci2011introduction}.

SRs baseados em filtragem colaborativa são os mais populares na área e vêm sido pesquisados há mais tempo. \cite{ricci2011introduction} 
É comum utilizar métodos baseados em vizinhança, nos quais um algoritmo de clusterização é usado para determinar grupos 
de usuários ou itens, tal como o algoritmo \textbf{KNN} (\textit{K-Nearest Neighbours}) \cite{da2018desenvolvimento}.

\subsubsection{Método Baseado em Conteúdo}

O método baseado em conteúdo parte da ideia de que usuários têm interesse em itens semelhantes àqueles que lhe foram uteis 
no passado \cite{ricci2011introduction}. No caso, é importante determinar a \textbf{semelhança entre itens} para então recomendar 
para determinado usuário itens semelhantes aos que foram \textbf{previamente bem avaliados} por ele. Nesse método 
é preciso estabelecer estratégias para descrever itens, bem como para montar o perfil dos usuários, descrevendo os tipos 
de itens que ele tem interesse. Após, deve ser feito o \textbf{comparativo} dos itens com o perfil do usuário para 
predizer seu interesse em tais itens. Geralmente procura-se dividir o universo dos itens, $I$, em categorias: relevantes ou 
irrelevantes, por exemplo. Para construir a classificação dos itens é possível usar uma série de algoritmos que 
realizam trabalho de \textbf{classificação estatística}, como por exemplo \textbf{árvores de decisão} \cite{pazzani2007content}. 

\subsubsection{Método Baseado em Confiança}

Conforme \cite{sinha2001comparing}, estudos indicam que os usuários têm a tendência de \textbf{valorizar mais as recomendações de amigos} do que aquelas feitas por outros usuários com perfil semelhante, porém desconhecidos e a qualidade das recomendações de amigos superam inclusive as feitas por sistemas de recomendação. A partir deste conceito, com a grande aderência de usuários 
à \textbf{redes sociais} um novo método para a construção de sistemas de recomendação está sendo estudado, trata-se do método 
baseado em confiança, ou sistema de recomendação social (\textit{social recommender system}) \cite{ricci2011introduction}.

A construção de SRs sociais depende do estabelecimento de uma \textbf{rede de confiança}, rede que descreve 
o nível de confiança entre seus membros. Assim, o usuário recebe recomendações de itens avaliados positivamente por usuários 
em sua rede de confiança. Estes SRs usam o conceito de \textbf{agregação e dissipação de confiança}, ou seja, 
dados um grupo de usuários $u_1  \dots u_n$, calcular o nível de confiança entre $u_1$ e $u_n$ considerando usuários intermediários
$u_2 \dots u_{n-1}$ que possuem alguma relação de confiança, direta ou indireta, com $u_1$ e $u_n$ (\textbf{dissipação}) ou 
combinar uma série de estimativas de confiança em um valor final (\textbf{agregação}) \cite{victor2011trust}.

Um ponto fraco de tais sistemas é que a recomendação é geralmente mais previsível e pode facilmente ser inundada por itens que o 
usuário já conhece, enquanto técnicas mais usuais de recomendação podem apresentar resultados mais inesperados, mas relevantes ao 
usuário \cite{sinha2001comparing}.

\subsubsection{Métodos Híbridos}

Métodos híbridos propõem a combinação de mais de um método de recomendação dentro de um sistema. É necessário para complementar 
técnicas que podem apresentar problemas em determinadas situações ou para oferecer resultados melhores aos usuários. Furlan 
\cite{da2018desenvolvimento} por exemplo combinou os métodos baseado em conteúdo e filtragem colaborativa para solucionar o problema da primeira 
avaliação. Já Massa \cite{massa2004trust} sugere que um método que leve em consideração a confiança entre usuários pode melhorar a 
performance de sistemas de filtragem colaborativa.

\subsection{Trabalhos Correlatos}

Os trabalhos correlatos foram escolhidos utilizando como critério a contemporaneidade e semelhança com o presente trabalho, 
de forma a trazer um embasamento atualizado das metodologias usadas para a resolução de problemas semelhantes.

\subsubsection{Desenvolvimento de um Sistema de Recomendação para Bibliotecas Digitais}

\cite{da2018desenvolvimento} aborda o problema da sobrecarga de informações dos pesquisadores baseando-se no perfil do currículo Lattes. 
O trabalho busca recomendações de produções científicas utilizando o motor de buscas Google Acadêmico e traz uma combinação 
das técnicas de filtragem colaborativa e baseado em conhecimento. A metodologia para gerar recomendações utilizada neste 
trabalho será usada no presente trabalho como referência para a elaboração do SR, levando em consideração os pontos fracos 
e fortes da abordagem descrita no trabalho. Em particular, será considerada a maneira com que o trabalho propôs solucionar 
o problema  da avaliação inicial de um SR de filtragem colaborativa através do método baseado em conteúdo.

\subsubsection{Técnicas de Recomendação para usuários de Bibliotecas
Digitais}

\cite{primo2006tecnicas} apresentam algumas das mais populares técnicas de recomendação, bem como a justificativa e contexto para a 
correta implementação dos mesmos. O trabalho descreve diversas abordagens para a elaboração de um SR de obras literárias 
em bibliotecas digitais, usando as técnicas de filtragem colaborativa e baseado em conteúdo bem como uma abordagem híbrida. 
O contexto do sistema de recomendação descrito no trabalho se assemelha ao do presente trabalho por ter como alvo uma 
biblioteca digital, sendo que as obras literárias da biblioteca podem ser comparadas aos artigos encontrados na plataforma 
Lattes. Além disto, o trabalho apresenta ainda um experimento para ilustrar a importância da opinião de especialistas.

O comparativo das metodologias usadas serve como referência para a elaboração do SR descrito no presente trabalho. 
As relações entre usuário e items (no caso, obras literárias) é descrita através de uma ontologia na qual conceitos são 
definidos pelos termos que os definem e organizados em uma hierarquia. A ontologia serve para descrever as relações entre 
item e usuário e serve como referência para a modelagem das relações do SR desenvolvido neste trabalho.

\subsubsection{Trust-aware Collaborative Filtering for Recommender Systems}

N artigo, \cite{massa2004trust} sugerem a possibilidade de melhorar as sugestões em sistemas de recomendação com métricas de confiança, descrevem a modelagem de uma rede de confiança e a necessidade de métricas de propagação de confiança, que consideram ser computável em 
mais usuários do que a similaridade de perfis. A métrica usada é a distância mínima entre nós para a estimativa de 
confiança local. Os autores sugerem ainda a aplicação do algoritmo PageRank \cite{page1999pagerank} como métrica de confiança global. Pretende-se seguir a arquitetura sugerida no trabalho para a construção de um SR que combina os métodos baseados em 
conteúdo e confiança, que é composta por módulos substituíveis que representam conceitualmente a aplicação de um algoritmo. A adaptação da arquitetura está descrita na seção \ref{sect:arch}.

\section{Materiais \& Métodos}

Para chegar nas recomendações, é proposto um trabalho em dois momentos: estimar a confiança entre pesquisadores e então
selecionar potenciais colaboradores baseando-se no perfil dos pesquisadores. A seleção será filtrada e ordenada de acordo com o 
nível de confiança estimado dos pesquisadores. O primeiro passo é modelar uma rede de confiança da comunidade científica, descrita 
por autores e publicações. Pode então ser feita uma seleção dos pesquisadores cadastrados com base no perfil do usuário alvo e 
ordenar a seleção de acordo com o nível de confiança de cada pesquisador. A confiança pode ser local ou global, sendo que local diz 
respeito à confiança estimada de um pesquisador específico em seus colegas e a global corresponde à a confiança da comunidade em 
cada pesquisador. Em termos de ferramentas para implementação, foi escolhida a linguagem Python, que possui riqueza de ferramentas 
e recursos para trabalhos relacionados a manipulação de dados e computação numérica.% \cite{python book} 

\subsection{Os Dados da Plataforma Lattes}

Os dados usados no trabalho são provenientes do banco de dados relacional da Plataforma Kennis (www.kennis.com.br), que extraí os dados dos pesquisadores dos currículos armazenados na Plaforma Lattes com o uso de um \textit{parser}, cuja versão inicial é descrita em \cite{prass2019parser}. Optou-se pelo uso dos dados da Plataforma Kennis e não pelos dados originais da Plataforma Lattes, pois durante o processo de \textit{parse} dos currículos, a Plataforma Kennis faz a limpeza e o pré-processamento dos dados, associando pesquisadores através das suas publicações em conjunto, conforme mostra a Figura \ref{fig:database}. A partir do banco descrito, é 
possível construir uma base de conhecimento focada especificamente no objetivo do presente trabalho, que é descrita a seguir.

\begin{center}
  \begin{figure}[ht]
    \centering
    \includegraphics[width=.8\textwidth]{database.png}
    \caption{Tabelas Pessoa, Produção e associativa \cite{prass2019parser}}
    \label{fig:database}
  \end{figure}
 \end{center}

Uma publicação científica, normalmente, é uma produção de algum \textbf{tipo} (artigo, livro, trabalho) que passou por um processo de 
revisão por pares e foi aceito como sendo uma contribuição válida, de autoria de um ou mais pesquisadores. A maioria das 
publicações da base de conhecimento possuem um conjunto de \textbf{palavras-chave}, informadas pelos autores, que servem como uma pista 
dos assuntos abordados. As publicações são feitas originalmente em um \textbf{idioma} e \textbf{país} específicos e 
obrigatoriamente possuem um \textbf{título} no idioma original bem como uma possível tradução para o inglês. O \textbf{ano} da 
publicação é considerado relevante devido à característica progressiva do conhecimento científico, onde observa-se constante 
introdução de novos dados e fatos. Em alguns casos é possível inferir a \textbf{abrangência} da publicação, por exemplo pode 
ser regional, nacional ou internacional. Algumas tuplas podem também conter a \textbf{natureza}, que deve ser vista como o 
nível de cobertura do assunto discutido alcançado pelo trabalho: completo, resumo e assim por diante. A ordem dos nomes dos 
autores geralmente é indicativo da importância do autor para a publicação: o \textbf{autor principal} geralmente é o primeiro 
nome e o \textbf{orientador} do trabalho o último. Entre eles, os colaboradores \textbf{intermediários}.

Os autores são o centro dos dados da base de conhecimento e são também responsáveis por cadastrar todas as informações lá encontradas. 
O perfil do pesquisador é composto por sua \textbf{formação}, \textbf{atuação profissional}, e \textbf{publicações} das quais fez 
parte. A formação é representada por um título, que diz respeito ao nível de ensino, por exemplo doutorado ou mestrado.

\subsection{Computando confiança} \label{sect:computing-trust}

Pode-se pensar na rede de confiança como sendo um grafo no qual os nodos são pesquisadores e as arestas publicações em conjunto. 
Uma colaboração em publicações é um voto de confiança entre os pesquisadores envolvidos. A matriz de adjacência pode ser usada 
para computar a propagação de confiança através da rede. A confiança estimada de determinado pesquisador pode levar em consideração 
o nível de confiança estimado dos pesquisadores que colaboraram com ele. Outro fator que pode ser considerado é a facilidade de 
colaboração, levando em  consideração “distâncias” na rede: se dois pesquisadores A e B têm laços de confiança com um intermediário 
C, a colaboração entre A e B tende a ser mais fácil do que se houvesse mais intermediários na rede. A figura \ref{fig:network} 
representa a rede de confiança descrita, os pesos das arestas são discutidos na secção \ref{sect:relevancy}.

\begin{center}
  \begin{figure}[ht]
    \centering
    \includegraphics[width=1\textwidth]{trust-network.png}
    \caption{Rede de confiança}
    \label{fig:network}
  \end{figure}
 \end{center}

O algoritmo PageRank \cite{page1999pagerank} foi inspirado em parte por estudos realizados em redes de citações acadêmicas, nas 
quais a relevância de um artigo era descrita por contagem de citações, por exemplo. Trata-se de um método para computar um \textit{ran\-king}  global de citações, pensado para computar a importância das páginas web. O \textit{ranking} $R$ de uma página é definido como a soma dos  \textit{rankings} das páginas que oferecem links para ela, ponderada pelo total de \textit{links} encontrados nas páginas.

Funciona da seguinte forma: é definido para cada página $u$ um conjunto $F_u$ de páginas as quais $u$ referencia e um conjunto $B_u$ de páginas que fazem referência à $u$. Sendo $\hat{A}$ a matriz de adjacência da web, tal que 

\begin{equation} \label{eqn:adjacency-matrix}
   \hat{A}_{i,j} =
    \begin{cases}
      1       & \quad \text{se } \text{ há links de i para j}\\
      0       & \quad \text{se } \text{ não há links de i para j}
    \end{cases}
\end{equation}

A matriz $A$ deve ser obtida dividindo todas as linhas de $\hat{A}$ por $|F_u|$ (o grau do nodo $u$). Assim, PageRank pode ser definido 
como $R = c(AR + E)$, sendo $c$ um fator de normalização.

Quando ocorrem ciclos no fluxo de referência, nos quais duas páginas se referenciam mutuamente e não fazem referência a nenhuma 
outra página, pode ocorrer o chamado \textit{rank sink}: referências exteriores injetam \textit{ranking} no ciclo, 
fazendo com que páginas do ciclo acumulem pontuação, porém sem distribuição. Para solucionar, foi introduzido o vetor $E$, que no modelo 
de PageRank é o conceito de um \textit{random surfer}, ou seja, uma probabilidade de um usuário da internet aleatoriamente mudar a 
página, sem seguir nenhum de seus \textit{links}. \cite{page1999pagerank}

A aplicação de PageRank à um grafo não-direcionado gera um vetor $R$ estatisticamente similar à distribuição de grau dos nodos da 
rede \cite{perra2008spectral}. Isto é, aplicando diretamente o algoritmo ao problema proposto, no final das contas a confiança 
seria proporcional ao número de publicações do autor (\textbf{centralidade} do nodo). Enquanto esta métrica é relevante, perde-se 
a ideia inicial: não é considerada a confiança dos colaboradores, somente o valor total de colaborações. 

Além disso, dois conceitos importantes não são levados em consideração: a relevância e o número de colaborações entre os 
pesquisadores. No caso da \textbf{relevância}, a importância da publicação é uma dica para o nível de confiança mútua entre os 
pesquisadores: colaborações em publicações importantes requerem maior confiança. O total de \textbf{colaborações em conjunto} 
entre um par de pesquisadores, por sua vez, indica uma relação mais duradoura, com mais confiança mútua. É importante distinguir 
total bruto de publicações de determinado pesquisador e o número de colaborações entre dois pesquisadores, pois há mais confiança 
quando observa-se frequentes colaborações.  Uma vez estabelecida uma heurística para a importância de determinada colaboração, 
pode-se somar as importâncias para avaliar a força dos laços de confiança.

\subsubsection{Relevância} \label{sect:relevancy}

As publicações não são iguais entre si. Para construir uma heurśtica que defina a relevância de uma publicação deve-se considerar as características descritas na representação da mesma, atribuindo pesos aos seus atributos. Aqui, a heurística considerada é a natureza da publicação, com pesos atribuidos conforme 
a tabela \ref{tab:relavancy}.

\begin{center}
  \begin{table}[ht]
    \centering
    \caption{Heurśtica de Relevância}
    \label{tab:relavancy}
    \includegraphics[width=.4\textwidth]{heuristics.png}
    \end{table}
\end{center}

Vale ressaltar que a heurística neste caso é relativa por haver muitos fatores que influenciam a relevância de uma publicação: 
artigos  em certas publicações de prestígio podem valer mais que capítulos de livro, e o mesmo pode ser verdade para textos em 
jornais ou revistas. Da mesma maneira, publicações mais recentes podem valer mais do que publicações mais antigas, porém o 
contrário pode ser verdade para linhas de pesquisa na área da história, por exemplo. É possível também considerar mais do que a 
relevância e quantidade das publicações para descrever os laços de confiança entre pesquisadores.

\subsubsection{Centralidade}

Considerando as heurísticas discutidas, é possível alcançar um fluxo de confiança mais interessante na rede modelada. O conceito 
de agregação e dissipação de confiança pode seguir a ideia do algoritmo \textit{PageRank} aplicando-se um algoritmo para o 
cálculo da centralidade dos nodos. No caso, a relevância das publicações e a quantidade de colaborações devem ser levadas em 
consideração na distribuição de confiança.

Várias métricas foram propostas na literatura para o cálculo da centralidade, em especial a apresentada em \cite{opsahl2010node} 
se encaixa particularmente bem no problema proposto por incorporar simultaneamente o grau (número de conexões) e a força 
(os pesos de cada conexão) dos nodos. Aqui, o peso pode ser a soma das relevâncias das obras publicadas em conjunto entre os 
pesquisadores. A fórmula proposta faz isso definindo um parâmetro $\alpha$ para ajustar a importância de grau e força:

\begin{equation} \label{eqn:centrality} 
 C_D ^{w \alpha} (i) = k_i \times \left( \frac {s_i} {k_i} \right) ^{\alpha} = k_i ^{1 - \alpha} \times s _i ^{\alpha}
\end{equation}

Ao incorporar o número e a relevância das publicações como um peso para as arestas da rede, é reintroduzido o conceito de 
considerar a confiança da comunidade nos colaboradores que depositaram confiança em determinado autor através de publicações 
em  conjunto para o cálculo da confiança estimada de tal autor. Assim, é possível chegar em um fluxo de confiança apurado 
\textbf{table me}
levando em conta informações sobre obras e autores que são relevantes para considerar a confiança compartilhada entre 
os membros da rede.

\subsubsection{Distância} \label{sect:distance}

A centralidade do pesquisador, ponderada pelo número de colaborações e suas relevâncias se mostra em teoria uma forte métrica de 
confiança global. O cálculo de confiança local, porém, oferece uma estimativa da \textbf{confiança subjetiva} de um usuário em 
relação aos membros da rede. Outra métrica valiosa neste contexto é a distância entre os pesquisadores, isto é, identificar 
\textbf{amigos de amigos} e pesquisadores próximos na rede é uma maneira de promover a colaboração entre pesquisadores: 
pesquisadores próximos na rede podem encontrar mais facilidade para realizar colaborações. Para tal, a relevância e quantidade de 
colaborações (pesos das arestas - confiança) deve ser um fator positivo e o número de nodos intermediários entre os autores um 
fator negativo (maior distância).

O algoritmo de Dijikstra \cite{dijkstra1959note} é definido para calcular distâncias em redes nas quais os pesos representam o 
custo de travessia. O trabalho de Opsahl e Skvoretz \cite{opsahl2010node} é definido em redes onde os pesos representam força 
dos laços, então os autores sugerem que os pesos devem ser invertidos. Além disso, o objetivo do trabalho é considerar também 
o número de nós intermediários, então os autores propõem novamente o uso de um parâmetro de ajuste, $\alpha$, que controla o quão 
importante considera-se o número de nodos intermediários e a força das conexões.

\begin{equation} \label{eqn:distance}
  d^{w\alpha}(i, j) = min \left( \frac{1}{ \left( w_{ih}^{\alpha} \right) } + \dots + \frac{1}{ \left( w_{hj}^{\alpha} \right) }  \right) 
\end{equation}

Esse conceito, aplicado à rede de colaborações, significa que a distância entre os pesquisadores levará em consideração o nível 
de confiança das conexões bem como o número de pesquisadores intermediários. Para $\alpha < 1$, caminhos com maior número de 
intermediários são considerados mais distantes, enquanto $\alpha > 1$ vai considerar mais importante a força das relações de 
confiança, atribuindo menores distâncias para caminhos onde há fortes relações de confiança entre os pesquisadores, podendo estes 
ter mais intermediários \cite{opsahl2010node}.

\begin{center}
  \begin{table}[ht]
    \caption{Distâncias}
    \label{tab:distances}
    \centering
    \begin{tabular}{|lcccc|}
      \hline
      \rowcolor[HTML]{343434} 
      \multicolumn{1}{|c}{\cellcolor[HTML]{343434}{\color[HTML]{FFFFFF} }}                       & \multicolumn{4}{c|}{\cellcolor[HTML]{343434}{\color[HTML]{FFFFFF} $ d^{w\alpha}(i, j) $}}                                               \\
      \rowcolor[HTML]{656565} 
      \multicolumn{1}{|c}{\multirow{-2}{*}{\cellcolor[HTML]{343434}{\color[HTML]{FFFFFF} Path}}} & {\color[HTML]{EFEFEF} $ \alpha = 0.00 $} & {\color[HTML]{EFEFEF} $ \alpha=0.50 $} & {\color[HTML]{EFEFEF} $ \alpha=1.00 $} & {\color[HTML]{EFEFEF} $ \alpha=1.50 $} \\ \cline{1-1}
      \multicolumn{1}{|l|}{\{A, B\}}                                                             & 1.00                        & 0.81                        & 0.50                        & 0.35                        \\
      \multicolumn{1}{|l|}{\{A, C, B\}}                                                          & 2.00                        & 1.53                        & 0.83                        & 0.54                        \\
      \multicolumn{1}{|l|}{\{A, D, E, B\}}                                                       & 3.00                        & 1.72                        & 0.47                        & 0.19                        \\ \hline
    \end{tabular}
  \end{table}  
\end{center}

\subsection{Pré-seleção de Recomendações} \label{sect:pre-selection}

Com as métricas de confiança aqui propostas, é possível estabelecer níveis de confiança da comunidade, uma rede subjetiva do 
pesquisador alvo ou distâncias ponderadas por confiança e usar as predições para oferecer sugestões de colaboradores para cada 
membro da rede em diferentes contextos. Todavia confiança apenas pode não ser suficiente para recomendações de qualidade. 
Considerar também a linha de pesquisa do pesquisador no momento e os perfis dos potenciais colaboradores pode aumentar a qualidade 
das recomendações: mesmo que a confiança em um pesquisador seja alta, a sugestão de colaboração pode não fazer sentido caso as 
linhas de pesquisa não se encaixem.

Para aprimorar as recomendações, é proposta uma pré-seleção dos itens utilizando um método baseado em conteúdo. A 
técnica consiste no cálculo de correspondência de palavras chave em um modelo de espaço vetorial (MEV), visto que 
esta é a mais comum em sistemas de recomendação baseados em conteúdo \cite{ricci2011introduction}. No MEV proposto, a ideia é estabelecer um 
perfil geral dos pesquisadores através de palavras chaves ponderadas por relevância, extraídas dos títulos e palavras chaves 
das publicações do qual o autor fez parte. O perfil do pesquisador é representado por um vetor em um espaço \textit{n-dimensional}: 
$d_j = {w_{1,j}, w_{2,j}, \dots ,d_{n,j}}$, no qual $w_{i,j}$ representa o quanto o termo $i$ é relevante dentro do trabalho do 
pesquisador $j$. Pode-se pensar em uma matriz na qual as linhas são pesquisadores, conforme descrito e as colunas representam os 
termos-chave extraídos do universo das publicações (\textit{corpus}), removendo palavras vazias - “ou”, “de”, “para” \dots - 
tanto em português quanto em inglês.

Tal matriz é construída através da técnica de vetorização \textit{TF-IDF}, na qual considera-se termos importantes aqueles que 
aparecem com frequência relacionados á um item específico e com menor frequência nos outros itens do \textit{corpus} 
\cite{pazzani2007content}. A partir disso é preciso computar a semelhança entre termos. Para tal, a proposta é o uso da 
similaridade de cossenos por ser a técnica mais comumente aplicada \cite{ricci2011introduction}. Para a pré-seleção dos itens, a 
proposta é basear-se em uma \textit{query} que representa o trabalho sendo desenvolvido pelo pesquisador no presente. A partir 
desta, pode-se calcular as similaridades dos perfis dos autores cadastrados com a \textit{query}, construindo assim a pré-seleção.

\subsection{Arquitetura} \label{sect:arch}

Até aqui foram descritas técnicas que podem ser usadas para a computação de confiança entre membros de uma comunidade, resultando 
em vetores de confiança global e local, bem como distâncias ponderadas por confiança. Também foi discutida a possibilidade de 
aprimorar recomendações baseadas em confiança usando o conteúdo dos itens disponíveis. Agora, detalha-se a descrição geral de como é 
possível combinar as metodologias descritas para obter um resultado. 

Em \cite{massa2004trust} é sugerida uma arquitetura de SR combinando filtragem colaborativa e método baseado em confiança. 
O sistema é descrito em módulos substituíveis e portanto pode ser usado para combinar os métodos baseado em conteúdo e em 
confiança. Basicamente, conforme apresentado na Figura \ref{fig:sr-arch}, a saída do método usado para pré-seleção é usada para filtrar e ordenar as recomendações a partir das confianças computadas.

\begin{center}
 \begin{figure}[ht]
   \includegraphics[width=1\textwidth]{SR-arch.png}
   \caption{Arquitetura do Sistema de Recomendação proposto}
   \label{fig:sr-arch}
 \end{figure}
\end{center}

É preciso então definir um método para oferecer ao usuário uma amostra seleta com os pesquisadores mais relevantes seguindo as métricas de confiança (filtragem por confiança) a partir de uma lista de itens ordenada por similaridade de cossenos e vetores de 
confiança estimada. Neste ponto, é possível usar alguma técnica para combinar as métricas de confiança, porém é importante 
as aplicar individualmente para observar os resultados. As linhas em pontilhado representam que os objetos são intercaláveis: 
pode-se usar um ou outro ou combinações. Após a pré-seleção, o custo computacional da aplicação das distâncias, por exemplo, 
diminui consideravelmente pois é preciso apenas computar distâncias entre o pesquisador alvo e uma amostra pequena de 
pesquisadores com o perfil compatível com a \textit{query}.

Portanto, uma vez aplicado o método por conteúdo, pode-se é aplicar a métrica de distâncias ponderadas por confiança em uma 
amostra reduzida da rede, visto que a métrica em questão em teoria é mais indicada para promover colaboração entre os membros da 
rede.

\subsection{Validação} \label{sect:validation}

Várias métricas são sugeridas para medir a performance de um SR quando a tarefa do sistema gira em torno de predizer avaliações, 
é possível realizar predições e calcular o erro diretamente, de maneira supervisionada. \cite{shani2011evaluating} sugerem 
que é importante considerar o contexto da aplicação de cada SR, pois como os objetivos de diferentes sistemas variam, é 
natural que variem também as métricas de validação. É preciso então avaliar cada caso e estabelecer quais as propriedades que 
influenciam o sucesso do SR.

Considerando que o principal fator de sucesso do SR aqui proposto é a promoção da colaboração entre os pesquisadores da rede de 
publicações, a métrica mais completa para a validação do sistema é subjetiva ao usuário: 

\begin{itemize}
  \item Estaria ele interessado em colaborar com o pesquisador recomendado ?
  \item Qual a facilidade de iniciar a colaboração ?
  \item Qual a confiança do usuário no pesquisador responsável ?  
\end{itemize}

Logo pode-se seguir por dois caminhos para a validação: implantar o sistema em alguma aplicação e acompanhar seus resultados no 
mundo real, hipótese na qual considera-se uma recomendação de sucesso aquela que resultar em colaborações, ou realizar 
recomendações para uma amostra seleta de pesquisadores e recolher seus \textit{feedbacks} sobre as recomendações.

No contexto da implementação proposta neste artigo, somente é possível a segunda opção. Portanto foi elaborado um questionário 
em dois momentos. A proposta é primeiramente perguntar a um grupo de pesquisadores qual a linha de pesquisa estão desenvolvendo, 
nas seguintes palavras: \textit{“No que você está trabalhando no momento?”} - A resposta para tal pergunta será usada como a 
\textit{query}, conforme ilustrado na figura \ref{fig:sr-arch}. A partir disso serão elaboradas as recomendações conforme a seção 
\ref{sect:arch}. A validação virá no segundo momento do questionário, no qual os pesquisadores serão apresentados às sugestões 
elaboradas e indagados sobre sua qualidade. É considerado que três perguntas são suficientes para a avaliação:

\begin{itemize}
  \item \textit{“Você acha relevante a colaboração com este pesquisador?”}
  \begin{itemize}
    \item Resposta: sim/não
  \end{itemize}
  \item \textit{“Em uma escala de um a dez, o quão fácil você considera realizar uma colaboração com este pesquisador?”}
  \begin{itemize}
    \item Resposta: Valor no interválo [1, 10], 1 representa facilidade mínima e 10 representa máxima facilidade
  \end{itemize}
  \item \textit{“Em uma escala de um a dez, qual o seu nível de confiança nesse pesquisador?”}
  \begin{itemize}
    \item Resposta: Valor no interválo [1, 10], 1 representa confiança mínima e 10 representa confiança total.  
  \end{itemize}
\end{itemize}

\section{Discussão \& Conclusão}

Aqui foi apresentado o conceito, no campo de sistemas de recomendação, de promover a colaboração entre membros de uma comunidade 
de pesquisadores. Foi introduzido o conceito de Sistema de Recomendação, as técnicas mais utilizadas e uma tendência mais recente, 
de recomendações baseadas em confiança. Sugeriu-se um modelo de rede de confiança na qual pesquisadores são conectados por  
publicações em conjunto, considerado um voto de confiança. Foram discutidas três métricas de propagação e agregação de confiança 
na rede, usando conceitos do algoritmo PageRank \cite{page1999pagerank} e de métricas de centralidade e distância entre nodos em grafos 
não-direcionados ponderados \cite{opsahl2010node}.

A partir de então, foi proposta a recomendação através da pré-seleção de perfis baseada em conteúdo seguida de filtragem dos perfis 
via distância ponderada por confiança, seguindo a arquitetura proposta por Massa e Avesani \cite{massa2004trust} para um SR híbrido 
utilizando o método baseado em conteúdo. Para a avaliação do modelo, foi apresentado o fator de sucesso que justifica a validação 
através de questionários respondidos pelo usuário mediante sugestões elaboradas pelo SR.

É possível estender o trabalho aqui proposto pensando em melhores heurísticas para a relevância das publicações, considerando por 
exemplo a data de publicação, com diferentes pesos para diferentes épocas. É possível também incorporar mais métodos discutidos 
na seção \ref{sect:computing-trust} para melhorar as recomendações, levando em consideração por exemplo métricas locais e globais, 
além das distâncias.

\bibliographystyle{sbc}
\bibliography{sbc-template}

\end{document}
